\documentclass[12pt,english,preprint]{aastex}
\usepackage[T1]{fontenc}
\usepackage[latin9]{inputenc}
\setcounter{tocdepth}{3}

\bibliographystyle{apj}
\usepackage{amstext}
\usepackage{amsmath}
\usepackage{amssymb}

\usepackage{graphicx}
\usepackage{color}
\usepackage{natbib}

\usepackage{xfrac}
\usepackage{diagbox}
\usepackage{multirow}
%%%%%%%%%%%%%%%%%%%%%%%%%%%%%% User specified LaTeX commands.




%%%%%%%%%%%%%%%%%%%%%%%%%%%%%% User specified LaTeX commands.
% \makeatother
\usepackage{babel}

\usepackage{xifthen}% provides \isempty test

\doublespace


\newcommand{\txt}{\text}
\newcommand{\mybf}{\bf}
\newcommand{\red}{\color{red}}
\newcommand{\su}[2]{#1_{\rm #2}}

\newcommand{\mdt}[1][]{ 
  \ifthenelse{\isempty{#1}}
  {\dot{M}_{\rm loc}}
  { {\dot M}_{\rm{loc},{#1}} } 
  } % call as \command[optional]
  
\newcommand{\Mdt}[1][]{ 
    \ifthenelse{\isempty{#1}}
    {\dot{M}_{}}
    { {\dot M}_{{#1}} } 
} % call as \command[optional]

\newcommand{\Rout}{\su{R}{out}}
\newcommand{\Rin}{\su{R}{in}}


\newcommand{\mdtcr}{\su{\dot{M}}{loc,cr} }

\newcommand{\MsolYrM}{ \,M_{\odot}{\rm yr}^{-1} }

\newcommand{\rsub}{ \su{R}{sub} }

\newcommand{\Tsub}{ \su{T}{sub} }
\newcommand{\Tc}{T}


\usepackage{babel}

% \makeatother

\usepackage{babel}
\begin{document}
\global\long\def\pd{\partial}%

\author{A. Dorodnitsyn\altaffilmark{1,2}, T. Kallman\altaffilmark{1}}
\altaffiltext{1}{Laboratory for High Energy Astrophysics, NASA Goddard Space 
Flight Center, Code 662, Greenbelt, 
MD, 20771, USA}
\altaffiltext{2}{University of Maryland, Baltimore County (UMBC/CRESST), Baltimore, MD 21250, USA}
% \altaffiltext{3}{Space Research Institute, 84/32, Profsoyuznaya st., Moscow, Russia}


\title{A physical model for radiative, convective dusty disk in AGN.}

\begin{abstract}
  
AGN accretion disk harbors and shields dust from external
illumination: at the mid-plane of the disk around a $M_{{\rm
BH}}=10^{7}M_{\odot}$ black hole, dust can exist 
at $0.1$pc from the black hole, compared to 0.5pc outside of the disk.  
We construct a physical model of a disk region 
approximately located  between the radius of dust sublimation at the
disk mid-plane and the radius at which dust sublimes at the disk
surface. Our main conclusion is that for a wide range of
model parameters such as local accretion rate and/or opacity, 
disk's own radiation pressure on dust changes its vertical structure
the disk becomes
highly convective. We discuss when such a disk transforms from geometrically thin to slim
and how our model fits into the narrative of a "failed wind" scenario of
\citep{CzernyHryniewicz11} and the "compact torus" scenario of
\citet{BaskinLaor2018MNRAS} incorporating them as variations of the radiative dusty disk model.
\end{abstract}


% \newcommand{\sub}[2][2]{(#2 + #3)^#1}

% A critical accretion rate above which such a disk is geometrically thick and the AGN is of type II, is determined by the Eddington critical accretion rate which defined with respect to dust opacity. 
% Our model predicts that the inner part of the RDR is convective which can explain 
% the origin of turbulence in the Broad Line Region.


% Such disk can be locally sub- or super-critical but globally sub Eddington with the excess mass recycled in global outflows. 
% The sub-critical RDR is geometrically thick 
% Convection in the inner part of the dusty
% disk can provide a natural 'turbulence driver' for the Broad Line
% Region and into the obscuring torus.  
% We found convection instability of the RDR is  
% be thin and convective or thick 
% For moderate accretion rates the inner part of the RDR
% Thus in our model it is the supply of gas from galactic scales which leads to RDR at
% $0.01-0.1$ pc for $10^7 M_\odot$ Black Hole.
% leads to weather disk transitioned from thin to geometrically thick and
% convective, or produces a dusty flow that is further hijacked by the
% radiation pressure from the nucleus.
% Our model is described by two parameters,
% the effective \textit{local} accretion rate and the ratio of the outer
% dust sublimation radius in the disk to dust sublimation radius derived
% from AGN luminosity.


% If the local mass-accretion rate exceeds  critical 
% : 1) convection cools
% the disk interior and allows dust to survive closer to the nucleus
% 2) radiation pressure puffs up the disk to $h/r\simeq0.1$, i.e. the
% disk is slim rather than geometrically thick as in the absence of
% convection 3) the disk ``bulge'' effectively interacts with external
% radiation and depending on the local accretion rate polar outflows
% of dust are possible. 4) 
% Active dust region can be relevant to observations of hot dusty "ring"
% in AGN 1068



\section{Introduction}

The radiative output of Active Galactic Nuclei (AGN) is powered
by accretion of gas and most of the gas potential energy is
released in the inner part of an accretion disk.  However, several crucial observational
characteristics of AGN are shaped considerably further away, at few$\times0.01-0.5$pc
from the Super-Massive Black Hole (BH).  It is also at approximately  this
distance from the BH the radiation flux is sufficiently diluted such
that dust grains can survive the illumination from the nucleus.
The presence of dust results in a $10-100$ -fold increase of opacity 
compared with only gas, and also to a  dramatic increase of coupling between the radiation
from the nucleus and gas.  

The radius outside of which dust can survive forms a dust sublimation
surface, a boundary between the inner, mostly
dust-free region and the outer part often associated with the dusty
torus.  The latter is
invoked to to explain the dichotomy between two types of AGN, in which 
optically thick equatorial material blocks the direct view onto the broad line region
and the accretion disk in type 2 galaxies
\citep{Rowan-Robinson77,Antonucci84,AntonucciMiller1985,UrryPadovani95}.

Mid-infrared (MIR) interferometric observations of nearby AGN 
clearly point to the presence of dust
\citep{Jaffe2004,Raban09,Tristram14,Tristram2011} 
at distances $\geq$ 0.1 pc from the center.  The relative numbers of type 
1 and type 2 objects suggest that the obscuring material is geometrically
thick.

The virial theorem predicts
that in order to be geometrically thick at a distance, $r\simeq1\,\text{pc}$,  
temperature of the obscuring gas should be of of the order of $10^{6}\text{K}$ for
a $10^{7}M_{\odot}$ BH. This is not compatible with survival of dust in the obscurer,
and hence is in conflict with the presence of dust inferred from IR observations.
On the other hand, the temperature expected at the surface of a thin accretion disk 
is $\leq$1000 K at $\sim$1 pc from the center.
% contradiction to observations of the copious amounts of dust. 
%Given the ubiquitousness of tori and the fact that standard accretion disk is very thin the potential mechanism should robustly
%explain how a thin disk "puffs up" to become a torus-like? 
Dust opacity provides a natural mechanism for making a dusty disk geometrically thick 
via radiation pressure.  
%Inside of the the sublimation surface, , dust sublimation rate exceeds dust formation rate and 
%the medium can become a strong line emitter as at the higher distances dust suppresses line 
%emission \citep{Netzer93,LaorDraine1993}. 
Reverberation mapping \citep[i.e.][]{Koshida2014} is generally consistent
 with putting the inner boundary of the torus to 
within the dust sublimation surface at $0.4-0.5$pc \citep{Kaspi2000}.
The location of the broad line region (BLR) relative to the center has been measured 
by reverberation to be $\sim$0.01 pc (ref??).
% Reverberation mapping RM, suggest that $\rsub$ marks the outermost 

Magnetic or/and radiation driving have been proposed as a mechanisms behind the formation of the BLR and the torus.
A line driven wind, i.e. a wind driven by the radiation pressure in UV lines \citep{ProgaKallman2004,Murr05}, 
has a launching radius which is $\sim 10x$ smaller, and correspondingly  resultant characteristic 
line widths which are a factor $\sim 10x$ greater than indicated by the maximum BLR line widths. A line driven wind from an accretion disk is not massive enough to be the torus. 

Magnetic fields, specifically large-scale magnetic fields, 
are an alternative or augmenting mechanism which can be the driving engine of the BLR and the torus. 
%Most models involving magnetic fields rely on self-similar solutions 
%\citep{Emmering92,Everett05,Keating12,KoniglKartje94} and thus intrinsically
%lack any spatial scale. 
Semi-analytical or numerical models \citep{Lovelace98,Dorodnitsyn16} 
show that large-scale B-field can support an AGN torus. 
%At pc scales matter is accreting near the mid-plane, within an accretion
%disk that is dense, cold and very geometrically thin: $h/R\propto T^{1/2}\simeq10^{-3}-10^{-4}$. 
Self-consistent numerical simulations of a 
thin disk threaded by the net vertical magnetic flux \citep{ZhuStone2018} 
show that thin disks cannot both transport large-scale magnetic fields 
\citep{LubowPapaloizou94,BKLovelace2000,BKLovelace2007} \emph{and} simultaneously have a 
massive, polar, MHD-driven (Magneto-hydrodynamic) outflow. 
In addition,  typical MHD flows have an approximate equipartition between magnetic energy density and 
gas energy. So the characteristic temperature of the gas near the launching radius of the magnetically-driven, 
gravitationally unbound outflow is also expected be of the order of the virial temperature.





% Similar radii are also associated with the
% size of the Broad Line Region (BLR). 
% from the nucleus is sufficiently diluted so that 
% 



% the observational the ubiquitous of type 2 AGN calls for a robust mechanism  
% which would shape their appearance presence of the




% It is perhaps not coincidental
% that approximately outside this spatial region,  the radiation flux
% from the nucleus is sufficiently diluted so that the dust formation
% rate exceeds that of dust sublimation.




% However the influence of dust on the inner workings of an AGN is not only in 
% One piece of argument 
% The above argument is  and a proxy of the luminosity of the nucleus .

% At the distances significantly greater than $r_{\rm sub}$ the accretion disk is very cold and correspondingly 
% geometrically thin that 

\citet{CzernyHryniewicz11}(hereafter CH11) suggested that the disk's own radiation may produce sufficient
radiation pressure on dust grains that would expel a failed wind from the outer accretion disk atmosphere. 
Such a failed wind then would be responsible for the formation of the broad line region seen
in Seyfert I galaxies. Developing this idea further, \citet{BaskinLaor2018MNRAS}
(BL18) concluded that the contribution from large graphite grains
near $T\simeq2000{\rm K}$ increases dust opacity which results in
a an inflated compact, torus-like structure near the observed BLR radius. 

In  a simple case, when dust is arranged in a spherically-symmetric
shell it cannot survive closer than dust sublimation radius: $R_{\text{sub}}\simeq0.2-0.5$
pc from a supermassive black hole.
%Anisotropic irradiation such as
%(\ref{eq:FluxAGNanisotr}) makes $r_{\text{sub}}(R,z)$
%{\mybf dependent on the angle of illumination of the incident radiation.}
If dust is contained in a cold and dense accretion disk it can survive
much closer to the BH, down to $\sim10^{-3}-10^{-2}$ pc (BL18, CH11). The radial
scaling of the effective surface temperature of the disk: $T_{{\rm eff}}\propto R^{-3/4}$
guaranties that beyond a certain radius, $R_{{\rm din}}$, $T$ drops
below the dust sublimation temperature
allowing survival of dust. This corresponds to 
to a dramatic increase of the opacity. The disk
radiation flux
%, $F_{{\rm loc}}$, calculated from (\ref{eq:DiskLocFlux})
may be strong enough to produce a non-negligible radiation pressure
$\propto(F/c)\,\kappa_{d}$, where $\kappa_{d}$ is dust opacity.
As  suggested by CH11 and  BL18 this can lead to the formation of failed
winds or produce a "compact torus" at $\text{few}\times10^{-2}$pc.
%In this paper we build a simple model of an accretion disk that
%takes into account the enhanced opacity associated with the formation
%of dust and associated changes to the physical properties of such a disk. 
In particular
there is a well defined region in a disk which spans
more than a decade in radius where dust has a major impact on the disk vertical
structure. The most important  impact is that such a disk is convectively
unstable. The latter changes disk internal structure and we discuss
how it may naturally provide a turbulent driver for the broad line
region connecting our model to the BLR model of BC11. 
%Finally we suggest
%that it is 
The internal radiation push 
%and external
%illumination which can build a path %to 
can provide an explanation of the geometrically
thick obscuration and avoids the  apparent paradoxes associated with 
gas pressure or turbulent vertical 
support.
%suggested by numerical simulations. %After all it is simple, no magic
%of the ``dramatic disk puffing up by external radiation'' is required.
%It is the disk itself that produces %the starter gas at the place where
%it is energetically and dynamically inevitable, which is then shaped
%by a much more powerful external flux.

Previous work has demonstrated the likely importance of dust in the dynamics of the gas in the 
torus and BLR.  However, there has not been an examination of the effects of dust on the internal structure of 
the accretion disk in the pc-scale region of AGN. 
The goal of this paper is to explore how radiation pressure on dust grains defines 
the vertical structure of the disk in the region where its temperature is comparable to $\Tsub$. 
%In previous works no such such calculations to the vertical disk structure were made. 
We will show that such changes are important, and lead to verifiable
results. A preview of the results is as follows:




% Our model thus incorporates a "failed wind" scenario of \citep{CzernyHryniewicz11}
% and the "compact torus" scenario of \citet{BaskinLaor2018MNRAS}
% as limiting cases.

% As in our previous papers, here we take the view and further investigate
% the idea that the ``obscuring torus'' is an integral part of a larger
% accretion flow. To provide a necessary covering fraction $h/r\simeq0.5$
% the otherwise very thin disk should either develop a ``bulge'' or
% form of an optically thick outflow. In particular it faces two characteristic
% challenges: 
% \begin{enumerate}
% \item 
% \item Formation problem: Application of the standard accretion disk theory
% predicts that accretion disk at $\sim1$pc from the SMBH is very geometrically
% thin, $h/r\sim10^{-2}-10^{-3}$ . The successful model should suggest
% a mechanism how to overcome this ``aspect ratio'' problem without
% contradicting point I. 
% \item 
% \end{enumerate}


% At $r\simeq1$pc, radiation flux from the
% nucleus, $F_{{\rm ext}}\gg F_{{\rm loc}}$, where $F_{{\rm loc}}$
% is the radiation flux generated locally in the disk. For that reason
% dust sublimation radius is often calculated with respect to $F_{{\rm ext}}$.
%  Notice, however that $F_{{\rm loc}}/F_{{\rm ext}}\propto1/r$
% i.e. the role of the local disk radiation flux is increasing as smaller
% radii. Consequently, at $r\simeq0.01-0.1$pc the pressure of the disk
% local radiation flux on dust should not be neglected. Some of the 

% As it is be shown in this paper that at $r\simeq10^{-2}-\text{a few}\times0.1$
% pc from the BH there exist a region where the interior disk temperature
% is smaller but comparable to dust sublimation temperature and this
% circumstance changes the disk structure, sometimes in significant
% ways. Our paper builds on two suggestions: 

% We adding the physical model for the disk which allows for several
% new predictions. This region is bounded by the geometrically thick
% torus on the outside and has . In general, convection makes the disk
% height smaller that it would be in a simple radiation-inflated model.
% On the other hand, our model predicts that since on the inside such
% disk is highly convective this convection is an attractive candidate
% to turbulently power the BLR.

% Radiation pressure on dust is an attractive mechanism for providing
% a dynamical coupling between the nucleus and the disk outskirts. Direct
% UV radiation pressure is the strongest due to the high UV-dust opacity
% but also completely dominated by the component in a radial direction.
% As it was seen from
% simulation of ... exposing a thin disk to external illumination does
% not readily produce a "puffed up" structure: If disk is thin the
% UV pressure is mostly down, not up. X-ray evaporated flow can produce
% a very hot flow that in favorable conditions can cool so dust forms
% and then be further pushed vertically through reprocessed near-IR
% pressure. The whole process was found to produce only temporary and
% erratic outbursts of obscuring material. In this paper we develop
% a model that relies on the disk ability to produce enough radiation
% pressure to make the disk locally thick or having a polar outflow
% of dust.

% Our paper has grown from two realizations:
% the accretion disk needs to provide internal mechanisms to become
% geometrically slim to intercept enough radiation flux and that enhanced
% dust opacity leads to dusty convection.  

\begin{itemize}
    \item  We first will develop an analytical model of an accretion disk, based on a modified 
    $\alpha$ disk approach, that includes  contributions from gas pressure 
    and radiation pressure on dust.   We show that there is a difference between the  
    radius of dust sublimation at the disk mid-plane and the radius at which dust 
    sublimes at the disk surface.  
%    In a broad range of parameters, dust opacity results in 
%    strong convection so that the energy transport along the vertical direction has significant
%    contribution from convection. 

    \item There is a region in the AGN accretion disk where the radiation pressure from the disk's own near- and mid-infrared 
    radiation shapes the vertical disk structure. In this paper we call this part of the disk the "Radiative Dusty Region" (RDR).

    \item It is well known that radiation pressure often leads to a strong convective instability both in stellar envelopes 
    and in radiative disks.   Developing a semi-analytical disk model,
    we calculate the parameter range where convection develops due to vertical radiation pressure on dust and calculate 
    the disk properties. In particular, we argue, that convective RDR is a powerful source of turbulence.

    \item The main parameter which determines the importance of local radiation pressure is the local accretion rate $\mdt$. 
    % We argue that were not fully appreciated 
    The disk becomes locally geometrically thick if the accretion is locally super-Eddington, with the latter calculated 
    with dust opacity which is 10-100 times larger than the typical opacity of gas without dust. 
    The latter allows to make an argument that the obscuration associated with type-2 AGN can be 
    attributed to a supper-critical dusty accretion in the RDR region, calculated in this paper.  
    % with recycled is lesser were not fully appreciated which if $\mdt\gtrs 1-10 M_\odot$   Our model predicts
    % \item \cite{BaskinLaor2018MNRAS} estimated the disk height at the innermost part of the disk where dust is permitted to exist  the value of $\mdt_\rm{ct } is the smallest $
\end{itemize}




\section{Properties of AGN at pc-scales}

In this section we examine the effect of dust formation or survival and its dependence on the global parameters of the 
AGN.   
In doing so, we adopt standard assumptions about accretion in a thin disk around a BH 
{\red I am not sure that there is a reference which I want to put here.There will be too
much emphasis on it, unless you have suggestions}.


\subsection{Global parameters }

It is customary to express the total luminosity of the AGN in terms of the accretion rate:


\begin{equation}
L=\epsilon\:\Mdt c^{2},\label{eq:AGNLum1}
\end{equation}

\noindent which is equivalent to the definition of an accretion efficiency, $\epsilon$ -
a parameter that is approximately bounded between $0.057$
for a non-rotating BH and $0.42$ for a BH rotating at maximum efficiency.
The mass-accretion rate, ${\dot{M}}$ in (\ref{eq:AGNLum1}) corresponds
to the innermost part of the disk where the most of the radiative output
is produced. After assuming $\epsilon$, it is standard to equate (\ref{eq:AGNLum1}) to the Eddington luminosity:



\begin{equation}
L_{{\rm E}}=\frac{4\pi cGM}{\kappa_{e}}=1.25\times10^{45}M_{7}\text{\text{\,erg} }\,{\rm s}^{-1},\label{eq:EddLum1}
\end{equation}

\noindent where $M$ is the mass of the BH, and to scale accretion rate in terms
of ``Eddington'' accretion rate:

\begin{equation}
\dot{M}_{{\rm E}}\simeq0.22\epsilon_{0.1}^{-1}M_{7}M_{\odot}{\rm yr}^{-1}\mbox{,}\label{eq:EddMdot}
\end{equation}


\noindent where in (\ref{eq:EddLum1}) and (\ref{eq:EddMdot}) the following
parameters are adopted: $\kappa_{e}=0.4\,{\rm cm^{2}g^{-1}}$ is the
Thomson opacity; we scale the BH mass in units of $M_{7}=M/10^{7}\text{M}_{\odot}$
and the efficiency of accretion $\epsilon=0.1$. 
%Hereafter, for scaled quantities we adopt 
%the notation $y/10^X \to y_X$. {\bf not really true}

The above picture is augmented with the  assumption that enough medium
is supplied to the AGN from galactic scales near the AGN outer
radius, $R_{{\rm AGN}}$. It is customary to define this as a radius
where the gravity from the BH dominates the gravitational field
of the host galaxy: 


\begin{equation}
R_{{\rm AGN}}=\frac{GM}{\sigma_{{\rm Blg}}^{2}}\simeq4.30\,M_{7}\left(\frac{\sigma_{{\rm Blg}}{\rm (km}{\rm s^{-1}})}{100}\right)^{-2}\text{ pc}=4.5\times10^{6}
\left(\frac{\sigma_{{\rm Blg}}{\rm (km}{\rm s^{-1}})}{100}\right)^{-2}\,{R_{g}},\label{eq:rAGN}
\end{equation}

\noindent where $\sigma_{{\rm Blg}}$ is the stellar velocity dispersion in
the bulge, and the last equality is given in terms of the gravitational
radius of the BH: 

\begin{equation}
R_{g}=\frac{2GM_{{\rm BH}}}{c^{2}}=2.95\times10^{12}M_{7}\,{\rm cm.}\label{eq:RadSchwarz}
\end{equation}

\noindent Hereafter, we reserve $R$ for the radius in physical units, and 
$r$ for the scaled radius: $r_y=R(\text{cm})/y(\text{pc})$.

A crude estimate for the vertical scale height of the disk at $R_{{\rm AGN}}$
is done assuming the scaling for the disk height: $H\sim v_{{\rm T}}/\Omega,$
where $h$ is the half-thickness of the disk, $\Omega$ is the orbital
velocity and $v_{{\rm T}}$ is the isothermal sound speed, $v_{{\rm T}}=\rho\su{{\cal R}}{gas}/\mu_m$,
and $\mu_m$ is the mean molecular weight and $\su{{\cal R}}{gas}$ is the gas
constant. Hereafter while calculating disk properties
we neglect the disk self-gravity (see Discussion), and 
from equation (\ref{eq:H2R_at_rAGN}) it follows that at $\su{R}{AGN}$ 
the disk is very thin:

% {\bf I changed to big R below}
\begin{equation}
H/R_{\text{AGN}}\simeq v_{{\rm T}}/(R\Omega)\simeq6.45\times10^{-3}\sigma_{{\rm Blg,}100}^{-1}T_{50}^{1/2}.\label{eq:H2R_at_rAGN}
\end{equation}
 
 \noindent Consequently, such a disk intercepts only a small fraction
of the radiation flux from the nucleus. 

It is instructive to review
radiative energy densities generated locally in the disk and compare
it to what is from external illumination. 
Radiation flux at pc-scales is dominated by the flux from the nucleus, $\su{F}{ext}$
which is produced in the 
inner parts of the disk. Its
angular dependence is the manifestation of the limb darkening effect of the disk, and has a simple
dependence on $\mu=\cos\theta$ where $\theta$ is 
the inclination angle from the normal to the disk
\citep[i.e.][]{sobolevCourseTheoreticalAstrophysics1975, 
sunyaevComptonizationLowfrequencyRadiation1985}:

\begin{equation}
F_{{\rm ext}}\simeq6\times10^{9}\mu(2\mu+1){\dot{M}}_{0.1}\epsilon_{0.1}\ r_{0.1}{}^{-2}\,\,{\rm erg}{\rm \,cm^{-2}{\rm \,s^{-1}}}\gg F_{{\rm loc}}\label{eq:FluxAGNanisotr}
\end{equation}

The local radiation flux generated in the disk, i.e. the normal flux at the disk's photosphere reads

\begin{equation}
F_{{\rm loc}}=\frac{3}{8\pi}\frac{GM}{r^{3}}\mdt\simeq3.4\times10^{4}
\,r_{0.1}^{-3}\mdt[0.1] M_{7}{\rm \,erg}\cdot{\rm cm^{-2}}\cdot{\rm s}^{-1}\mbox{,}\label{eq:DiskLocFlux}
\end{equation}

\noindent and it follows that until matter can spiral down to a fraction of a parsec, the release of the
gravitational potential energy produces local radiative output that
is negligible for the gas dynamics. Radiation flux, $\su{F}{ext}$ depends on the mass-accretion rate in the inner disk, 
while $F_{{\rm loc}}$ depends on local accretion rate, and in the following we reserve $\Mdt$ for global, and $\mdt$ for local 
accretion rates.

If a thin, cold, dusty disk is illuminated by the UV flux $F_\txt{ext}$, 
due to very high UV opacity the radiation is stopped immediately near the surface of the disk
heating dust to approximately $\Tsub$. Assuming all incoming radiation is converted to IR it follows 
that the temperature in such an idealized cold disk is {\it decreasing} towards the equatorial plane.
The radiation pressure, ${\bf g}_\txt{rad}\propto \nabla T$ points downwards and 
is balanced by the vertical gradient of the gas pressure at the characteristic density: 

\begin{equation}
\su{n}{eq} =n(P_g=P_r) \simeq 6.12 \times 10^{10}\,T_{1500}^{3}\,{\rm cm^{-3}}\mbox{,}
\label{eq:densPgas=Prad}    
\end{equation}

\noindent where $T_{1500}$ is the dust temperature scaled in units of $\Tsub=1500$K, also $P_g$ is the gas pressure:

\begin{equation}
P_{g} = \rho{\cal R}T\mbox{,}\label{eq:Pgas}\\
\end{equation}
and $P_r$ is the radiation pressure:  
\begin{equation}
P_r  =  aT^{4}/3\mbox{.}\label{eq:Prad}
\end{equation}

\noindent where $a$ is the radiation constant, and to simplify notation such as in (\ref{eq:Pgas}), the mean molecular weight,
$\mu_m$ is absorbed in the definition of the gas constant,
${\cal R}=\su{{\cal R}}{gas}/\mu_m$, and in the rest of the
paper we adopt $\mu_{{\rm m}}=1$.  Even when radiation pressure on dust is important in the bulk of the disk, at mid-plane the density exceeds the density in equation \ref{eq:densPgas=Prad}, 
$n_c\gg \su{n}{eq}$,  so despite the enhanced opacity due to dust the mid-plane pressure is dominated by $P_{g}$.



Accretion in a thin disk far  from
a BH is slow. The free-fall time-scale is the shortest in the hierarchy of time-scales: 
$\displaystyle t_{{\rm dyn}}=1.49\times10^{2}r_{0.1}^{3/2}{\rm yr}$. 
% \label{eq:tdyn}
However, the accretion time-scale corresponds to the viscous time-scale $t_{{\rm visc}}$ in the disk:

\begin{equation}
  t_{{\rm a}}\propto t_{{\rm visc}}=\frac{R}{v_{r}}=
  5.15\times10^{7}r_{0.1}^{1/2}T_{1500}^{-1}\alpha_{0.1}^{-1}\,\text{yr},\label{eq:t_visc}
\end{equation}

\noindent where $\alpha<1$ is the effective viscosity parameter, introduced
by \cite{ShakuraDiskModelGasAccretionRelativistic1972},
and $\su{\alpha}{0.1} = \alpha/0.1$. 
A geometrically thin disk cools efficiently through
radiative losses and as the disk cooling-time is much smaller
than $t_{a}$: 
$\displaystyle t_{{\rm th}}=1/(\alpha\,\Omega)=1.492\times10^{3}\alpha_{0.1}^{-1}r_{0.1}^{3/2}{\rm yr}$,
it is  approximately $\su{t}{dyn}\lesssim \su{t}{th}$.

From \eqref{eq:t_visc} it is clear that a buildup of matter (in a thin disk) at large radii 
cannot quickly propagate through the disk towards smaller radii. 
Hereafter we allow the situation in which the local accretion rate exceeds the 
rate at the center, $\mdt\gtrsim\Mdt$. 
The local accretion rate $\mdt$ defines the rate of energy production in the disk, and 
correspondingly the local vertical radiation flux, $\su{F}{loc}$ in \eqref{eq:DiskLocFlux}.



% Correspondingly \emph{local} accretion rate can be significantly different than that
% in the inner parts of the disk and $\dot{M}=const.$
% within a pc-scale AGN disk may actually not be accurate.
% A word of caution is that given a great AGN dynamical range, local
% accretion rate, $\dot{M}$ need not be the same as $\dot{M}_{a}$:
% $\dot{M}\gtrsim\dot{M}_{a}$ implying that accretion disk winds remove
% the excess material. 
% What can lead to Besides of the natural variation in
% mass-supply rate from the galaxy, disk instabilities such as self-gravitating,
% non-axisymmetric instability can lead to strong variation of $\dot{M}$
% can readily develop in AGN disk on the dynamical time-scale (\ref{eq:tdyn})
% leading to the effective disconnect between $\dot{M}_{{\rm a}}$ and
% $\dot{M}$ in (\ref{eq:AGNLum1}).




\subsection{Disk thickness When Radiation Pressure is Important}

Assuming a thin disk
approximation the equation for vertical balance reads:

\begin{equation}
\frac{dP_g}{dz}=-\rho\,g_{z} + \rho\frac{\su{F}{loc}\kappa}{c}\mbox{,}\label{eq:dPdz}
\end{equation}

\noindent where $\kappa$ is the opacity of the accreting material which
is assumed to be comparable to that of dust.  The gravitational 
acceleration, $g_{z}$ is found from:

\begin{equation}
g_{z}=\Omega^{2}z\mbox{,}\label{eq:gz}
\end{equation}

\noindent The disk, throughout this paper, is assumed to be Keplerian, $\Omega=\Omega_{K}$, where

\begin{equation}
\Omega_{K}=(G\su{M}{BH}\,R^{-3})^{1/2}\mbox{.}\label{eq:OmegaKepler}
\end{equation}


In the case when  
radiation pressure dominates, the characteristic scale-height of the disk follows from \eqref{eq:dPdz} after neglecting the contribution from the gas
pressure: 

\begin{equation}
H/r\simeq\frac{3}{2}\frac{\mdt{}}{\mdtcr}
\simeq8\times10^{-3}\kappa_{10}\,\mdt[0.1]\,r_{0.01}^{-1}\mbox{,}\label{eq:H2R_rad}
\end{equation}

\noindent where $\mdtcr$ is the Eddington accretion rate, calculated using 
dust opacity $\kappa_{d}$. 

\begin{equation}
\mdtcr=\frac{4\pi cr}{\kappa_{d}}\simeq12.3\,\kappa_{10}^{-1}
\,r_{0.01}\,M_{\odot}{\rm yr}^{-1}\mbox{,}\label{eq:MdotCritDust}
\end{equation}

\noindent where $\kappa_x=\kappa/x$ is the opacity scaled in $x({\rm cm^2\,g^{-1}})$ units of opacity, 
and $\mdt[0.1]$ is the local accretion rate scaled
in $0.1 M_{\odot}{\rm yr}^{-1}$. In 
models taking into account the contribution from large graphite grains (BL18) 
the critical mass-accretion rate is correspondingly smaller:
$\mdtcr = 2.5\,\kappa_{50}^{-1} M_{\odot}{\rm yr}^{-1}$. If the local accretion rate exceeds $\mdtcr$ then the disk scale height becomes comparable to the radius.
It is instructive to compare (\ref{eq:MdotCritDust})
with the critical value of $\dot{M}$ calculated assuming electron scattering
opacity, $\dot{M}_{\text{E}}\simeq 0.2M_{\odot}{\rm yr}^{-1}$ for the same set of 
parameters. For the disk not to be super-Eddington globally, the excess mass of the
gas $\sim {\rm few}\MsolYrM$ should be expelled via winds along the way towards the BH.


% \begin{equation}
% \mu=\frac{\mdt}{\mdtcr}\label{eq:muParamNonDimMdot}
% \end{equation}
% with $\mu\simeq0.6$ corresponding to $H/r=1$. It turns out that
%  However, dust opacity increases
% the effective optical depth in (\ref{eq:TcTauc}). Notice that equilibrium
% between $P_{g}$ and $\Pi$ is obtained at characteristic density:

One can notice that
the gas density in the disk mid-plane is much higher than required by 
(\ref{eq:densPgas=Prad}), so when internal radiation pressure
of the disk is negligible the gas pressure from the disk 
can balance radiation pressure due to external radiation at plausible levels.
If the thick disk is supported by the radiation pressure, the condition
$H/R\propto1$ can be recast as a consequence of the Virial theorem
for the disk temperature, namely $T\simeq T_{{\rm vir,r}}$ where
$T_{{\rm vir,r}}$ is the ``virial'' temperature for the radiation
dominated medium \citep{Dorodnitsyn11a}:

\begin{equation}
T_{{\rm vir,r}}=\left(\frac{GM}{a R}\right)^{1/4}
\simeq 1755.93 \left(\frac{M_{7}(n/10^7)}{r_{0.1}}\right)^{1/4}{\rm \,K}\mbox{,}
\label{eq:TvirRad}
\end{equation}

\noindent derived for a spherically symmetric shell. 


\subsection{Dust Sublimation Region in a Disk}\label{sec:DustSubRegIndisk}



\section{The Inner and Outer Dust Sublimation Scales}

In order to calculate the structure of a thin disk,  the
assumption was made in (SS73) that viscous dissipation is proportional
to the gas density $\rho$.  Here we assume that all the dissipated energy is transported vertically
by radiation. This gives the following estimate for the temperature
at the disk surface:

\begin{equation}
T_s=879\;\left(\frac{M_{7}\mdt[0.1]}{r_{0.01}^{3}}\right)^{1/4}\text{\,K}\label{eq:Teff}
\end{equation}

\noindent where it was assumed that $T_{s}=T(\tau_{{\rm phot}}=2/3)$ in which
$\tau_{{\rm phot}}$ is the optical depth at the disk photosphere.
When $R=10^{-3}{\rm pc}$ the surface temperature reaches 
$1479$K and little dust can survive in the disk.

If external illumination is neglected, 
the vertical decrease of temperature
within a disk ensures that the mid-plane temperature, $T_c$
is always greater than the surface temperature $T_{s}$. Approximately, 
$T_{c}$ is a factor of $\tau_{c}^{1/4}$ greater than $T_{s}$
, where $\tau_{c}$ is the vertical optical depth of the disk (see Section \ref{sec:Solution}). 
In general $\tau_{c}$ should be calculated from the solution for
the vertical structure of the disk. From standard gas pressure only
$\alpha-$disk solution we get

\begin{equation}
T_c=\frac{1}{2}\left(\frac{3}{2}\right)^{1/5}\kappa^{1/5}J(R)^{2/5}\mu_m^{1/5}
\text{\ensuremath{ \mdt }}^{2/5}\pi^{-\frac{2}{5}}\Omega^{3/5}{\cal R}^{-1/5}
\alpha^{-\frac{1}{5}}\su{\sigma}{B}^{-\frac{1}{5}}\simeq2\times10^{3}\kappa_{10}^{1/5}M_{7}^{3/10}
\text{\ensuremath{\mdt[0.1]}}^{2/5}r_{0.1}^{-9/10}{\rm K}\mbox{,}\label{eq:TcAnalytPgas}
\end{equation}
where the factor $J(R)$ is related to the inner boundary condition near the BH and for sub-parsec distances
it is $J\simeq1$ to a good accuracy, and $\su{\sigma}{B}= ac/4$ is the Stefan-Boltzmann constant.

Figure \ref{fig:TSurfVsTcVsTsubSketch} shows graphs of
$T_s(R)$ from \eqref{eq:Teff} and $T_c(R)$ from 
\eqref{eq:TcAnalytPgas}. Two intersections of these two curves with the
$\Tsub$ line define the inner, $\su{R}{in}$ and outer, $\su{R}{out}$ dust sublimation
radii in a disk.
The inner sublimation radius
is defined as the radius where the disk surface temperature equals 
dust sublimation temperature:

\begin{equation}
\su{R}{in}\simeq5\times10^{-3}M_{7}^{1/3}\mdt[0.1]^{1/3}\su{T}{sub,1500}
^{-4/3}{\rm pc}\simeq5\times10^{3}M_{7}^{-2/3}\mdt[0.1]^{1/3}\su{T}{sub,1500}^{-4/3}\,r_g.\label{eq:Rin}
\end{equation}

\noindent where $\su{T}{sub,1500}$ is the dust sublimation temperature in units of 1500K.  
Similarly, from (\ref{eq:TcAnalytPgas}), we define an outer sublimation radius, $R_{{\rm out}}$
as such a radius that at $r\le \Rout$ dust sublimes at the mid-plane:



\begin{figure} 
\includegraphics{RadDiskTempSketch}
\caption{Schematics of the dusty region in a disk}
\label{fig:TSurfVsTcVsTsubSketch}
\end{figure}
  


\begin{equation}
\su{R}{out}\simeq0.14M_{7}^{1/3}\kappa_{10}^{2/9}
\mdt[0.1]^{4/9}
\alpha_{0.1}{}^{-\frac{2}{9}}\su{T}{sub,1500}^{-10/9}{\rm pc}\simeq
1.5\times10^{5}\mdt[0.1]^{4/9}\kappa_{10}^{2/9}M_{7}{}^
{-\frac{2}{3}}\alpha_{0.1}{}^{-\frac{2}{9}}\su{T}{sub,1500}^{-10/9}\,r_g\mbox{,}\label{eq:Rout}
\end{equation}

\noindent Since $\su{R}{out}$ is found from $T(\Sigma_c) = \Tsub$, 
the disk's column density, $\Sigma_c = 2\rho_c H$  
should found from the solution for the disk.

As the gas spirals to $r\le \Rout$, $\Tc$ becomes greater
than dust sublimation temperature $T_{{\rm sub}}$. Dust is destroyed
first near the mid-plane and then, as $r$ gets smaller, progressively
above. Thus $R_{{\rm out}}>R_{{\rm in}}$ which  defines a region
within the disk where hot dust exists at some height in the disk. Notice that at $R=R_{{\rm out}}$
disk surface is quite cool: $T_{{\rm s}}\simeq156\left(M_{7}\dot{M}_{0.1}\right){}^{1/4}r_{0.1}{}^{-3/4}{\rm K}$.
Such a dramatic difference between $T_s$ and $T_c$ is, of course, the 
result from the blanketing effect from the disk.
Dependence of $\su{R}{in}$ and $\su{R}{out}$ on $\su{M}{BH}$ follows 
if we assume the scaling $\mdt\propto\Mdt\propto \su{L}{Edd} \propto \su{M}{BH}$:

\begin{equation}
\su{R}{in}\propto \su{M}{BH}^{2/3} \mbox{,}
\end{equation}
and
\begin{equation}
\su{R}{out}\propto \su{M}{BH}^{7/9}\mbox{.}
\end{equation}

\noindent Increasing the mass of the BH, we obtain $R_{{\rm in}}\simeq0.1$pc , and $R_{{\rm out}}\simeq5$pc 
for $M_{{\rm BH}}=10^{9}M_{\odot}$.


The accretion rate near the BH $\Mdt$ determines the central luminosity  $L$ through \eqref{eq:AGNLum1} and thus the global dust sublimation radius $\rsub$ 
(see Section \ref{sec:DustSubRegIndisk}). 

\begin{equation}\label{eq:RsubAGN}
  \su{R}{sub}=0.13\left(\frac{f(\theta)\epsilon_{0.1}{\dot{M}}_{0.1}}{T_{1500}^{4}}\right)^{1/2}\,\text{pc}\mbox{,}
\end{equation}

\noindent where $f(\theta)=\mu(2\mu+1)$ is the angular dependence of the radiation flux from the nucleus, i.e. \eqref{eq:FextFormula}. 
%In general $\su{R}{sub}$ should be found from a solution of a non-linear equation,
The result is the dust sublimation surface, which, generally speaking, is different from a simple spherically symmetric case such as 
\eqref{eq:RsubAGN} which was calculated for $f=1$. 
A difference between our definition of $\su{R}{out}$
in \eqref{eq:Rout} and the definition of BL18 is that the latter
define $\su{R}{out}$ as the dust sublimation radius for AGN 
\eqref{eq:RsubAGN}
As long as energy is transported vertically via radiation, it follows
from (\ref{eq:Rout}) and \eqref{eq:Rin} that $\Rout\propto\mdt[0.1]^{1/9}\Rin$.
Another interesting scaling: $\su{R}{sub}/R_{{\rm out}}\sim (\Mdt/\mdt)^{1/2}$ 
follows from \eqref{eq:Rout}, \eqref{eq:RsubAGN}.

Without the shielding protection of the accretion disk the fate of
dust above such disk depends on whether it is inside or outside the
AGN dust sublimation surface $\rsub(\theta)$:

\begin{enumerate}
  \item if $\rsub \gtrsim \su{R}{out}$, the dust above the disk does not survive.
  \item if $\su{R}{in} \lesssim \rsub \lesssim \su{R}{out}$, depending on $\mdt$ the disk can 
  be 1) thin or 2)thick/outflowing, 
  Figure \ref{fig:DiskSketch}.  
\end{enumerate}

{\bf Here we need a discussion:  under what conditions 
(mdot, M, etc.) do we get each of these cases??}



\begin{figure}
  % \includegraphics[scale=0.7]{DiskSketch} 
  \caption{\label{fig:DiskSketch} 
  Convective dusty disk with an outflow.
  When $\mdt\gtrsim \Mdt  (L_{{\rm bol}})$ (i.e accretion is locally supercritical)
  disk's own radiation pressure pushes dust above the disk. 
  At  $R>\su{R}{sub}^{\rm AGN}$ disk outflow is launched by the disk's 
  own IR pressure, with a possibility to being further accelerated by the radiation
  pressure from the nucleus.
  Not to scale.}
  \end{figure}
  
%More accurate predictions are beyond what our simple model can deliver and that can only be addressed via 
%full radiation-hydrodynamics modeling.

% \begin{equation}
% q=\frac{R_{{\rm sub}}^{{\rm AGN}}}{r_{{\rm out}}}\simeq(0.6\kappa_{50}^{-2/9}
% -0.9\kappa_{10}^{-2/9})\times\alpha_{0.1}^{2/9}(f(\theta)\epsilon_{0.1}\Mdt[0.1])^{1/2}
% \su{\dot{m}}{loc,0.1}^{-4/9} 
% M_{7}^{-1/3}T_{1500}^{-8/9}\mbox{,}\label{eq:qParam}
% \end{equation}
% where $f(\theta)$ represents the dependence of the flux from the
% nucleus on the inclination c.f. \eqref{eq:FluxAGNanisotr}.


\section{Solution for the Disk Vertical Structure}\label{sec:Solution}
\subsection{Basic equations}

In this section we describe the details of our model used to derive estimates 
in the previous sections. 
We adopt the $\alpha-$disk theory of SS73 and include pressure of infrared (IR) radiation on dust in the disk 
interior. The radiation is produced by viscous dissipation and assumed to be in thermodynamic 
equilibrium with gas-dust mixture. Dust particles are assumed to be fully coupled to the gas. 

The equation of state is that of a mixture of ideal gas and radiation:

\begin{equation}
P=P_g+P_r\mbox{,}\label{eq:Ptot}
\end{equation}
where $P$ is the total pressure, $P_r$ is the radiation pressure defined 
in \eqref{eq:Prad}, and  $P_{g}$ is the gas pressure defined in \eqref{eq:Pgas}.
Analytic solutions for the
limiting cases of $P_g \ll P_r$, and $P_g \gg P_r$ are derived in
\citep{ShakuraSunyaev73}.

We assume that in the RDR region, accretion proceeds near the equatorial plane:

\begin{equation}
\dot{M}=2\pi r v\,\Sigma\mbox{,}\label{eq:MdotEq}
\end{equation}

\noindent where to simplify notation in this section $\dot{M}=\mdt$ is the time averaged accretion rate, $v$ is the radial
gas velocity and $\Sigma$ is the surface density: 

\begin{equation}
\Sigma=\int_H^H\rho\,dz\simeq2 H\,\rho\mbox{,}\label{eq:SigmaS}
\end{equation}

\noindent where $H$ is the vertical scale-height of the disk and $\rho=\rho(z=0)$.
i.e. in a single-layer disk model, all height-dependent quantities are taken at the
disk mid-plane.

\begin{equation}
\nu\Sigma=\frac{1}{3\pi}\dot{M}\mbox{,}\label{eq:AngMom}
\end{equation}

\noindent where we neglected a factor, related to the inner boundary condition,
(i.e. $J(R)$ in \eqref{eq:TcAnalytPgas}).
The viscosity is proportional to the total pressure, $P$:

\begin{equation}
  \nu=\alpha\,\frac{P}{\text{\ensuremath{\Omega\rho}}}\mbox{.}
  \label{eq:defNuAlphaPtot}    
\end{equation}

\noindent The only relevant component of the radiation flux in a pc-scale thin disk
is vertical. 
The vertical flux $F$ satisfies: 

\begin{equation}
\frac{\partial F}{\partial z}=\frac{9}{4}\rho_c
\,q_{{\rm v}}\equiv\frac{9}{4}\Omega^{2}\rho_c\nu\mbox{,}\label{eq:dFdz}
\end{equation}

\noindent where $q_{{\rm v}}$$({\rm ergs\cdot s^{-1}\cdot g^{-1}})$ is the
specific rate of viscous energy dissipation: 

\begin{equation}
\su{q}{v}=\Omega^{2}\nu\mbox{,}\label{eq:qv(specific)}
\end{equation}

\noindent and then integrating equation (\ref{eq:dFdz}) between $-H$ and
$+H$, one gets the vertical radiation flux from the disk:

\begin{equation}
F^{+}=\frac{3}{8\pi}\dot{M}\Omega^{2}\mbox{,}\label{eq:RadFlux}
\end{equation}

\noindent An implicit assumption was made when integrating \eqref{eq:dFdz} to obtain \eqref{eq:RadFlux}. That is,
after equation \eqref{eq:dFdz} was rewritten as
${\displaystyle \frac{\partial F}{\partial\sigma}=\frac{9}{4}q_{{\rm v}}},$
where ${\displaystyle \sigma(z)=\int_{0}^{z}\rho\,dz}$ is the mass
coordinate, and then integrated over height, it was assumed a constant rate of \emph{specific} viscous dissipation, $\su{q}{v}$. 

The radiation moment equation in
a plane-parallel case valid for the geometrically thin disk is:

\begin{equation}
\frac{\partial E}{\partial\tau}=3\frac{F}{c}\mbox{,}\label{eq:dEdTau}
\end{equation}

\noindent where $E$ and $F$ are radiation energy density and radiation flux
and $\tau$ is the vertical optical depth. 

The boundary condition for \eqref{eq:dEdTau} is $E(\tau=0)=2F^+/c$. 
Integrating \eqref{eq:dEdTau} results in $E=(3F^+/c)(\tau+2/3)$ and

\begin{equation}\label{eq:SigmaT4c}
  \su{\sigma}{B}T^4 = \frac{27}{64}\nu\kappa\,\Omega^2\Sigma^2\mbox{,} 
\end{equation} 

\noindent introducing the subscript "c" for mid-plane quantities, 
from (\ref{eq:SigmaT4c}) it follows that if $\tau_{c}\gg1$, 
where 
${\displaystyle
\tau_c = \int_{0}^{H}\kappa \, \rho \,dz\simeq \Sigma \, \kappa_d }$
then (\ref{eq:TcAnalytPgas}) gives: 

\begin{equation}
T_c\propto\tau_{c}^{1/4}T_{{\rm s}}\mbox{,}\label{eq:TcTauc}
\end{equation}


\subsection{Solution with radiation pressure }\label{sec:SolWithRadPressure}
% Even when $P_g$  always dominates over $P_r$ at the mid-plane of the disk above 
% it the input from the radiation pressure quickly increases.


% As it was shown above, temperature of the disk surface is approximately
% $\tau_{c}^{1/4}$ times smaller than at the mid-plane. 

The solution for $\Tc$ for gas-dominated disk
was given in (\ref{eq:TcAnalytPgas}). Here our goal is to
calculate the same for the {\it average} properties of the disk which is 
supported by an arbitrary combination
of gas and radiation pressure: $P=P_{{\rm gas}}+aT^{4}/3$. 
% When calculating viscosity in \eqref{eq:nuAlphaPtot} we take into 
% account that 
The total pressure, $P$ is 


\begin{equation}
P=\rho\left({\cal R}T+\frac{2aT^{4}H}{3\Sigma}\right)\mbox{,}\label{eq:nuAlphaPtot}
\end{equation}

\noindent and also that

\begin{equation}
{\displaystyle \rho=\frac{\Sigma}{2H}}\label{eq:rhoEqSigmaOver2H}\mbox{.}
\end{equation}
% from (\ref{eq:SigmaS}) and 

\noindent The approximate relation for the scale-height follows from
the equation for the vertical balance:
$\displaystyle \frac{dP}{dz}=-\rho\,g_{z}$,
which is equivalent to \eqref{eq:dPdz}, after noticing that in diffusion approximation 
$\displaystyle \su{F}{loc}=-c/(\kappa\rho)dP_r/dz$:
% {\bf what is $\Pi$????  }

  
\begin{equation}
{\displaystyle H^{2}=\frac{P}{\Omega^{2}\rho}\mbox{,}}\label{eq:ScaleHeight}
\end{equation}
Equation \eqref{eq:ScaleHeight} is the equivalent of the vertical balance equation.
After much algebra it can be shown that in this case the solution should be calculated from 
a non-linear algebraic equation for $\Tc$.

\begin{equation}
\left(1-\frac{3a\alpha c{\cal R}\Omega}{4\left(F^{+}\right)^{2}\kappa}
\Tc^{5}\right)^{2}-\frac{3a\alpha\kappa}{4c\Omega}\Tc^{4}=0\mbox{.}
\label{eq:TcNonLinEq1}
\end{equation}

At a given $r$, with $F^{+}$ calculated from (\ref{eq:RadFlux})
and with $\kappa$ from (\ref{eq:kappaLogistic}) and $\Omega$ from
(\ref{eq:OmegaKepler}), equation (\ref{eq:TcNonLinEq1}) is solved
numerically. 
To ensure numerical stability when solving \eqref{eq:RadFlux}
we assume that above $T_{{\rm sub}}$ the opacity switches from $\kappa_{d}+\kappa_{e}$ to
$\kappa_{e}$  and approximate $\kappa(T)$ by a bridging formula:

\begin{equation}
\kappa(T)=\frac{\kappa_d}
{\exp\left(
  \frac{T-\su{T}{sub}}{\Delta T}
  \right)+1} + \kappa_e
\mbox{,}
\label{eq:kappaLogistic}
\end{equation}


\noindent where the bridging parameter $\Delta T$ is fixed at $\Delta T=0.1$ as well as $\kappa_{d}=10\,\text{cm}^{2}\text{g}^{-1}$.
To find roots of equation \eqref{eq:TcNonLinEq1} the following
procedure is adopted:
A first approximation for $T_0$
is found after $\kappa(T_0)$ is estimated from (\ref{eq:kappaLogistic}), 
where $T_0$ obtained
for the gas-pressure only solution \eqref{eq:TcAnalytPgas}.
Then $T_0$ is used as initial guess to numerically 
solve (\ref{eq:TcNonLinEq1}) with (\ref{eq:kappaLogistic}) 
for the true value of $T$.
Multiple roots of equation \eqref{eq:TcNonLinEq1} should be weeded out via checking that
they produce positive right-hand-side in the equation (\ref{eq:ScaleHeight}).


Once $\Tc$ is known the surface
density $\Sigma$ is found form equations 
(\ref{eq:AngMom}),\eqref{eq:nuAlphaPtot} and (\ref{eq:Ptot}),
and (\ref{eq:rhoEqSigmaOver2H}):


\begin{equation}
\Sigma=\frac{64\pi\sigma}{9 \kappa} 
\frac{\Tc^{4}}{\dot{M}\Omega^{2}}
\mbox{.}\label{eq:SigmaSolArbP}
\end{equation}

\noindent The result of a numerical solution of the equation 
\eqref{eq:TcNonLinEq1}
with respect to $\Tc$ is shown in 
Figure (\ref{fig:RadDiskTempPlot}) where 
disk model for $P=P_g$ is compared with the model for
$P=P_g+P_r$.


\begin{figure}
  \includegraphics[scale=0.7]{RadDiskTemp} 
  \caption{
  Disk temperature at the disk mid-plane.  
  Black: convective.  Blue: gas+radiation pressure, 
  equation \eqref{eq:TcNonLinEq1}.
  Orange: gas-pressure-only,
  equation \eqref{eq:TcAnalytPgas}. 
  The intersection with the dashed line, $T_{{\rm sub}}=1500$K marks the 
  position of $\Rout$.}
  \label{fig:RadDiskTempPlot} 
  \end{figure}
  

While this figure more accurately illustrates the relation between
$\Rin$ and $\Rout$, we still probably underestimate $\Rout$
because $\kappa_d$ becomes important before $T$ is approaching
$T_{{\rm sub}}$. Thus, the region where the dust opacity changes the
vertical structure of the disk is extended further away. From
\eqref{eq:Rin} and \eqref{eq:Rout} one can see that the size of this
region relatively weakly depends on $\dot{M}$.

\newcommand{\Rie}[1]{\su{R}{in}  = #1}
\newcommand{\Rio}[1]{\su{R}{out} = #1}

\begin{table}[h!]
\centering
\begin{tabular}{|c|c|c|c|c|c|}
  \hline
  {\diagbox{$\frac{\dot{M}}{M_{\odot}\text{yr}^{-1}}$}
  {$\frac{\su{M}{BH}}{M_\odot}$}}
  & $10^{5}$ &  $10^{6}$ 
  & $10^{7}$ & $10^{8}$ & $10^{9}$\\
  \hline
      0.01 & $5.53\times 10^{-3}$ & $7.29\times 10^{-3}$ 
       & $8.61\times 10^{-3}$ & $9.57\times 10^{-3}$ & $1.96\times 10^{-2}$\\ 
      0.01 & $1.09\times 10^{-2}$ & $1.81\times10^{-2}$ 
      & $7.17\times 10^{-2}$ & $1.54\times 10^{-1}$ & $3.33\times 10^{-1}$\\ 
  \hline
      0.1 & $7.29\times 10^{-3}$ & $8.61\times 10^{-3}$ 
      & $9.57\times 10^{-3}$ & $1.02\times 10^{-2}$ & $2.04\times 10^{-2}$\\ 
      0.1 & $4.15\times 10^{-2}$ & $8.94\times 10^{-2}$ 
      & $1.93\times 10^{-1}$ & $4.15\times 10^{-1}$ & $8.94\times 10^{-1}$\\ 
  \hline            
  1 & $8.61\times 10^{-3}$ & $9.57\times 10^{-3}$ 
    & $1.02\times 10^{-2}$ & $2.25\times 10^{-2}$ & $4.83\times 10^{-2}$\\ 
  1 & $1.06\times 10^{-1}$ & $2.28\times 10^{-1}$ 
    & $4.9\times 10^{-1}$ & $1.06$  & $2.28$  \\
  \hline            
    2 & $1.72\times 10^{-2}$ & $9.79\times 10^{-3}$ 
      & $1.04\times 10^{-2}$ & $2.84\times 10^{-2}$ & $6.18\times 10^{-2}$\\ 
    2 & $1.37\times 10^{-1}$ & $2.94\times 10^{-1}$ 
               & $6.34\times 10^{-1}$ & $1.36$ & $2.91$\\
  \hline            
    10 & $9.57\times 10^{-3}$ & $1.02\times 10^{-2}$ 
     & $2.25\times 10^{-2}$ & $4.91\times 10^{-2}$ & $1.06\times 10^{-1}$\\ 
    10 & $2.21\times 10^{-1}$ & $4.76\times 10^{-1}$ 
    & $1.03$ & $1.86$ & $3.83$\\             
  \hline
  \end{tabular}
  \caption{$\Rin$ and $\Rout$ calculated from the numerical solution 
  of equation \eqref{eq:TcNonLinEq1}. Left column: accretion rate, 
  $\mdt$. Upper raw: mass of the Black Hole, $\su{M}{BH}$. 
  Each cell not belonging to the first raw or the first column
  contains two numbers: upper number: $\Rin$; lower number: $\Rout$.
  }
  \label{table:1}
\end{table}



                         
% \begin{tabular}{|l|c|c|c|c|}
%   \hline
%   {\diagbox{$\frac{\dot{M}}{M_{\odot}\text{yr}^{-1}}$}
%   {$\frac{\su{M}{BH}}{M_\odot}$} 
%   }& $10^{5}M_{\odot}$ & 
%   $10^{6}M_{\odot}$ & $10^{7}M_{\odot}$ & $10^{8}M_{\odot}$\\
%   \hline
  % \multirow{2}{0.5in}{0.01} & $5.53\times 10^{-3}$ & $7.29\times 10^{-3}$ 
  %               & $8.61\times 10^{-3}$ & $9.57\times 10^{-3}$ \\ 
  %              & $1.09\times 10^{-2}$ & $1.81\times10^{-2}$ 
  %              & $7.17\times 10^{-2}$ & $1.54\times 10^{-1}$\\ 
  % \multirow{2}{0.7in}{0.1} & $7.29\times 10^{-3}$ & $8.61\times 10^{-3}$ 
  %              & $9.57\times 10^{-3}$ & $1.02\times 10^{-2}$ \\ 
  %              & $4.15\times 10^{-2}$ & $8.94\times 10^{-2}$ 
  %              & $1.93\times 10^{-1}$ & $4.15\times 10^{-1}$\\             
  % \multirow{2}{0.7in}{1} & $8.61\times 10^{-3}$ & $9.57\times 10^{-3}$ 
  %              & $1.02\times 10^{-2}$ & $2.25\times 10^{-2}$ \\ 
  %              & $1.06\times 10^{-1}$ & $2.28\times 10^{-1}$ 
  %              & $4.9\times 10^{-1}$ & $1.06$\\
  % \multirow{2}{0.7in}{2} & $1.72\times 10^{-2}$ & $9.79\times 10^{-3}$ 
  %              & $1.04\times 10^{-2}$ & $2.84\times 10^{-2}$ \\ 
  %              & $1.37\times 10^{-1}$ & $2.94\times 10^{-1}$ 
  %              & $6.34\times 10^{-1}$ & $1.36$ \\
  % \multirow{2}{0.7in}{10} & $9.57\times 10^{-3}$ & $1.02\times 10^{-2}$ 
  %              & $2.25\times 10^{-2}$ & $4.91\times 10^{-2}$ \\ 
  %              & $2.21\times 10^{-1}$ & $4.76\times 10^{-1}$ 
  %              & $1.03$ & $1.86$\\             
  % \hline
  % \end{tabular}

% \begin{tabular}{|c|c|c|c|c|}
% \hline
% $\frac{M_{{\rm BH}}}{M_{\odot}}$ & $10^{5}M_{\odot}$ & 
% $10^{6}M_{\odot}$ & $10^{7}M_{\odot}$ & $10^{8}M_{\odot}$\\
% \hline
% % $\frac{\dot{M}}{M_{\odot}\text{yr}^{-1}}$ & & & & \\
% \hline
% \multirow{2}{0.5in}{$\frac{\dot{M}}{M_{\odot}\text{yr}^{-1}}$} & 1 & 2 & 3 & 4 \\ 
%        & 5 & 6 & 7 & 8\\ 
% \multirow{2}{0.5in}{0.01} & 1 & 2 & 3 & 4 \\ 
%        & 5 & 6 & 7 & 8\\ 
% \end{tabular}

% \begin{tabular}{|c|c|c|}
%   \hline
%   \multirow{2}{0.5in}{A} & 1 & 2\\ 
%    & 3 & 4\\
% \end{tabular}
  

{\bf I think there should be a table of Rin and Rout vs Mdot or something more quantitative to show the results of your solution of the nonlinear equation}

\section{Convection}

If a gas element which is rising over a small distance, adiabatically
and in pressure equilibrium with the environment, is found to 
be lighter when contrasted to its surroundings, then
buoyancy force will keep propelling it further \citep{KippenhahnWeigert94}.
This corresponds to the  Schwarzschild criterion for convective
stability:

\begin{equation}
\left|\frac{dT}{dz}\right|_{{\rm ad}}>\left|\frac{dT}{dz}\right|_{{\rm rad}}\mbox{.}\label{eq:SchwarzschildCrit}
\end{equation}


In this section we revisit assumptions about vertical transport of
energy in the disk 
and show, that as soon as the mid-plane 
temperature exceeds the sublimation temperature, $\Tc \sim T_{{\rm sub}}$, there is a certain range of radii, 
$r_{{\rm in}}<r<r_{{\rm out}}$, where the disk is convectively unstable. There are two potential reasons why
convection is expected to be important: 1) The regular convection
associated with Schwarzschild criterion for a radiative, dusty disk. 2) Strong height-dependence of the opacity associated with the onset of dust opacity at some height above the midplane.


% Correspondingly,
% if $(dT/dz)_{\text{ad}}>(dT/dz)_{\text{rad}}$ the medium is unstable
% (remember that $dT/dz$ is negative). 
% In radiative equilibrium, the
% vertical temperature gradient is found from the diffusion approximation:
% \begin{equation}
% {\displaystyle \frac{dT}{dz}\propto-\frac{\kappa\rho}{T^{3}}F_{z}}\label{eq:dTdzPropTo}
% \end{equation}
% Increase of $\kappa$ associated with dust formation, (\ref{eq:kappaLogistic})
% moves inequality (\ref{eq:SchwarzschildCrit}) further toward instability.
The result is analogous to convective instability of a radiation-dominated 
part of a standard $\alpha$-disk. 
In a disk in radiative equilibrium, the entropy, $S_r$, falls abruptly with increasing $z$ and
convection drives the equilibrium towards an isentropic
state: $S_{r}\propto const.$ \citep{BKBlinn77}, where $S_{r}$ is
the entropy of the gas+radiation when $P_r\gg P_{g}$:

\begin{equation}
S_{r}=\frac{4}{3}\frac{aT^{3}}{\rho}\mbox{.}\label{eq:EntropyRad}
\end{equation}

The second reason is that in a disk in radiative equilibrium 
$T$ decreases from the mid-plane towards higher $z$.
A dramatic increase of the radiation pressure associated
with such an opacity jump is expected. 
% It is reasonable to think that most of
% dissipation is happening near the equatorial plane and the vertical
% radiation flux, $F^{+}\propto const.$ 
As long as $T_{c}>T_{{\rm sub}}$,
there exists $z_{s}(r)$ within a vertical slice of a disk where there
is a transition from dust-free opacity, $\kappa_{m}$ to dust opacity,
$\kappa_{d}$. Correspondingly $dT/dz$ becomes very large at $z_{s}$, 
triggering convection, which works towards smoothing the vertical distribution 
entropy.

The convection establishes a new distribution of $\rho$ and $T$
so as to decrease $dT/dz$. 
Our simplified model discussed so far adopts vertically averaged
quantities and a 
more elaborate treatment of convection calls for more
sophisticated methods. In this drvyion  we  estimate the efficiency of convection, treating it
as a perturbation and adopting the solution for the disk in radiative equilibrium  as an initial condition.



\subsection{Convective region}

To estimate the convective flux we adopt the fiducial values of parameters:
${\displaystyle R=R(T= \Tsub)= 0.1}$ pc,
temperature at $T_{{\rm sub}}=1500$K i.e. corresponding to the
situation of the dust sublimation in the disk mid-plane. 
Estimating $dT/dz$ for
$n = 1.84\times10^{11}T_{1500}^{3}\,\text{cm}^{-3}$
corresponding to $P_{{\rm g}}=P_r$ case \eqref{eq:densPgas=Prad}, we have
 
\begin{equation}
\frac{dT}{dz}\propto\frac{T_{{\rm s}}-T_{c}}{H}\simeq-1.14\times10^{-12}\:{\rm K}\cdot{\rm cm}^{-1}\mbox{,}\label{eq:dTdz_Model_num}
\end{equation}

\noindent where $T_{s}$ and $T_{c}=T(z=0)$ are estimated from the disk model.
This should be compared with the adiabatic gradient 
$\left(\frac{dT}{dz}\right)_{{\rm ad}}$ estimated from:

\begin{equation}
\left(\frac{dT}{dz}\right)_{{\rm ad}}=
-g_{z}\rho\left(\frac{\pd T}{\pd P}\right)_{s}=-g_{z}\rho H
\simeq-2\times10^{-12}\:{\rm K}\cdot{\rm cm}^{-1}\mbox{\mbox{,}}\label{eq:dTdz_ad_num}
\end{equation}

\noindent where the disk height $H$ is calculated taking into account the equation
of state (\ref{eq:Ptot}). From
(\ref{eq:dTdz_Model_num}) and (\ref{eq:dTdz_ad_num}) it follows that
in the region of $\Rin\lesssim r\lesssim \Rout$ adiabatic
$\left(\frac{dT}{dz}\right)_{{\rm ad}}$ can have similar magnitude
as $\frac{dT}{dz}$ warranting further investigation. 
% Column density,
% $\Sigma_{c}$ and the density $\rho_{c}$ is found from equations
% (\ref{eq:SigmaSolArbP}) and (\ref{eq:rhoEqSigmaOver2H}).

For the mixture of gas and radiation, the convective flux is \citep{KippenhahnWeigert94,BisnovatyiKogan2001book}:

\begin{equation}
F_{{\rm conv}}=\frac{1}{4}l^{2}\rho\,C_{p}\left(\frac{g_{z}}{T}\right)^{1/2}(\Delta\nabla T)^{3/2}\sqrt{\frac{4P_r}{P_{g}}+1}\mbox{,}\label{eq:Fconv}
\end{equation}

\noindent where $l$ is the mixing length, $C_{p}$ is the heat capacity at constant pressure for the mixture
of gas and radiation:

\begin{equation}
C_{p}={\cal R}\left(\left(\frac{4P_r}{P_{g}}\right)^{2}+\frac{20P_r}{P_{g}}+\frac{5}{2}\right)\mbox{.}\label{eq:Cp}
\end{equation}

\noindent The temperature excess of the convective element over its surroundings
is represented by $(\Delta\nabla T)$ which is found from:

\begin{equation}
\Delta\nabla T=\left(\frac{\pd T}{\pd P}\right)_{s}\frac{dP}{dz}-\frac{dT}{dz}=-g_{z}\rho\left(\frac{\pd T}{\pd P}\right)_{s}-\frac{dT}{dz}\mbox{,}\label{eq:DelNablaT}
\end{equation}

\noindent We adopt $l\simeq\epsilon_{0}H$ for the mixing length, where
$H$ is the half-thickness of the disk and $\epsilon_{0}\leq1$ is
the mixing length parameter for which we adopt the value $\epsilon_{0}=0.1$.
Adopting  $T$ and $\rho$ from the disk model for the disk model from Section (\ref{sec:SolWithRadPressure}), we numerically 
calculate $\su{F}{conv}$ from \eqref{eq:Fconv} and \eqref{eq:DelNablaT}.
The result is shown in 
Figure \ref{fig:ConvRegion},

{\red this figure should have axis labels}
{\red the locations of Rin Rout and Rsub should be identified on these panels}
where non-dimensional
$\su{\left(\frac{dT}{dz}\right)}{ad}$ and $\left(\frac{dT}{dz}\right)$ are plotted. Consider the situation when $r$ is decreasing
(i.e. tracing the diagram from right-to-left): the curves cross when $\vert\su{\left(\frac{dT}{dz}\right)}{ad}\vert>\vert\frac{dT}{dz}\vert$ and 
the medium becomes convectively unstable. 
In the left column the effect of accretion rate is shown: for $\mdt=0.1\MsolYrM$ the RDR is convective at $\Rin<r<0.13$pc;
increasing the accretion rate pushes the convective region further out: for $\mdt=2\MsolYrM$, the convective region is at $\Rin<r<0.3$pc.
In the right column one can see a similar effect if the mass of the BH is increased to $\su{M}{BH}=10^9M_\odot$.

\begin{figure}
\includegraphics[scale=1.5]{gradTAdiabVsMod} 
\caption{Transition to convection at the disk mid-plane. Shown are non-dimensional
(in code units) gradients: $\left(\frac{dT}{dz}\right)_{{\rm ad}}$
-orange line; actual $\frac{dT}{dz}$ - blue line. Horizontal axis:
distance from the BH in pc. Shaded area: region of convective instability.}
\label{fig:ConvRegion}
\end{figure}




% In order to estimate convective flux at any given point in the atmosphere
% of an accretion disk, the radiation field consists from the remote
% contribution from the nucleus, $F_{{\rm ext}}$ and local contribution
% generated within a local patch accretion disk $F^{+}$, the latter
% is assumed to be directed perpendicular to the disk plane. Flux $F_{{\rm ext}}$
% is very sensitive to the (unknown) attenuation at smaller radii.



% To estimate the effect of convection we consider only vertically
% averaged properties of the disk. In such as disk, the vertical
% distribution of temperature, $T$ can be found if it is assumed that
% the viscous heating is proportional to the density, $\rho$ (for limitations
% see Discussion):

% \[
% T=T_{c}(1-(z/z_{s})^{2})^{1/4}\mbox{,}
% \]
% where $z_{s}$ is the scale-height of the disk. If $P_{r}\gg P_{g}$
% such a disk is vertically homogeneous ($\rho\sim const)$. is decreasing
% vertically when $\,z$ increases. It is typically assumed that such
% a fluid element completely blends with the surrounding medium after
% it travels over a characteristic ``mixing length'', $l$.

\section{Effects of external irradiation}

In presence of illumination by the radiation flux, boundary condition
for the surface temperature of the disk, $T_{s}$ should be modified:

\begin{equation}
\sigma_{{\rm B}}T_{s}^{4}=F^{+}+f(1-A)F_{n}\mbox{,}\label{eq:BCwithIrradiation}
\end{equation}

\noindent where $F_{n}$ is the component of the external flux normal to the
surface of the disk, $A$ is the disk albedo, $f$ the attenuation and angular-dependent
factor and $F^{+}$ is found from (\ref{eq:RadFlux}). Solving radiation
transfer equation (\ref{eq:dEdTau}) with (\ref{eq:BCwithIrradiation})
we have

\begin{equation}
\sigma_{{\rm B}}T^{4}=\frac{3}{4}(\tau+\frac{2}{3})F^{+}+F_{{\rm ext}}\mbox{,}\label{eq:TsolWithIllum}
\end{equation}

\noindent where $F_{{\rm ext}}=f(1-A)F_{n}$ and the photosphere is placed at
$\tau=2/3$. From (\ref{eq:TsolWithIllum}) one can see that if $F^{+}\tau_{c}\gg F_{n}$,
the mid-plane temperature, $T_{c}$ is practically independent from
external sources of heating {[}LyutSyun{]}. The total attenuation
factor in (\ref{eq:BCwithIrradiation}) is approximately $0.5e^{-2/3}\simeq1/4$
for $A\simeq0.5$.

Vertical structure of the disk is calculated from (\ref{eq:dFdz}), and
(\eqref{eq:dEdTau}, yielding vertical distribution of temperature
in the illuminated disk:

\begin{equation}
T=T_{c}\left(1-\frac{3}{2}\frac{\tau_{c}F^{+}}{T_{c}^{4}}\left(\frac{\sigma}{\sigma_{c}}\right)^{2}\right)\mbox{,}\label{eq:TcWithIllum}
\end{equation}

\noindent where we assumed the specific rate of viscous energy dissipation to
be constant, $\tau_{c}\simeq\sigma_{c}\kappa_{d}$. Surface temperature
as found from (\ref{eq:BCwithIrradiation}) can, on the other hand,
be significantly influenced by the illumination.

\begin{equation}
F_{{\rm ext}}\simeq\delta(1-A)\frac{L}{4\pi r^{2}}\,\mu(2\mu+1)\mbox{,}\label{eq:FextFormula}
\end{equation}

\noindent where $L$ is calculated from (\ref{eq:AGNLum1}) and the factor $\delta\simeq H/r$
is related to the angle between the surface of the disk and the direction
of radiation flux \citep{meyerVerticalStructureAccretion1982,Spruit96}.
Estimating $\delta$ from (\ref{eq:H2R_rad}):

\begin{equation}
\delta\simeq\frac{3}{8\pi}\frac{\kappa_{d}}{cR}\dot{M}\mbox{.}\label{eq:delt_illum}
\end{equation}

\noindent the effect from the external flux is demonstrated in Figure \ref{Fig:TcWithExtIrrad}
which shows $T_{c}$ and $T_{s}$ for $A=0.5$. Not surprisingly, $T_{s}$
is noticeably influenced by irradiation. 

\begin{figure}
\includegraphics{diskTsurfIluum_1x2pan}\caption{Effect of the illumination on the surface temperature of the disk.}
\label{Fig:TcWithExtIrrad} 
\end{figure}


In case of the central radiation flux, the dominating component of radiation force is directed very 
nearly along the disk surface.  The vertical radiation pressure in a single-scattering approximation, scales in the same way as the gravitational
force. Thus illumination of a thin disk by the UV flux is unlikely responsible for "puffing up" of the AGN disk at pc-scales.
X-rays penetrate much deeper, and can 
potentially lead to a much stronger 
"puffing up" due to IR pressure \citep{Chang_etal07,Dorodnitsyn16}.


The predictive power of simple calculations of the effect of the irradiation
is limited. The attenuation of the radiation flux from depends on
the obscuring properties of winds at smaller radii and can be addressed
only via global numerical modeling.

\subsection{Dust above the disk}

The unattenuated UV radiation is promptly stopped in a very thin layer
where it will be further converted into IR as well as heating the
gas. The UV opacity of the $0.1{\rm \mu m}$ grain is about its geometrical
cross-section, $\kappa_{{\rm UV}}\simeq10^{2}f_{{\rm d,\,0.01}}\,{\rm cm^{2}g^{-1}}$,
where $f_{{\rm d,\,0.01}}$ is gas to dust mass ration in $10^{-2}$.
The thickness of such a ``photospheric'' layer is

\begin{equation}
\delta l_{{\rm UV}}/r_{{\rm sub}}^{{\rm UV}}\simeq5\times10^{-2}f{\rm d}_{0.01}n_{5}^{-1}L_{46}^{-1/2}\mbox{,}
\label{ConversionLayerUV}
\end{equation}
while the penetration length of
X-rays is significantly higher:

\begin{equation}
\delta l_{{\rm XR}}/\su{r}{sub}^{\rm UV}\simeq\begin{cases}
0.023, & 0.5<E<7keV\\
0.014, & E>7{\rm keV}
\end{cases}\times n_{5}^{-1}L_{46}^{-1/2},\label{ConversionLayerXRay}
\end{equation}

where $\kappa_{{\rm XR}}$ is the X-ray opacity consisting of photoionization
and Compton cross sections. In general $\delta l_{{\rm XR}}\simeq\frac{\kappa_{{\rm UV}}}{\kappa_{XR}}\delta_{{\rm UV}}.$
We adopt the photo-ionization cross-section from \citep{Maloney96}: for
$0.5<E<7$ keV we have $\sigma_{{\rm XR}}\simeq2.6\times10^{-22}\text{cm}^{2}$
and for $E>7{\rm keV}$: $\sigma_{{\rm XR}}\simeq4.4\times10^{-22}\text{cm}^{2}$.

% It is instructive to note that the  
% radiation pressure
% at $\Tsub=1400-2000{\rm K}$ is 
% corresponds to $n_{{\rm rg}}\simeq10^{11}{\rm cm}^{-3}$. 
{
Within the UV conversion layer
the UV radiation exerts pressure
of the order of $P_r\simeq0.03\, {\rm dyn\,cm^{-2}}$
on the dusty surface of the disk/torus.  
If, from within the disk, such a layer is supported 
entirely by the gas pressure, then, recalling that the equilibrium density at 
$P_r=P_g$, is  $n \simeq10^{11}{\rm cm}^{-3}$ \eqref{eq:densPgas=Prad}, 
the UV penetration length can be estimated to be just $10^{-8}-10^{-7} \su{r}{sub}^{\rm UV}$.

Illumination by
an X-ray flux has a devastating effect on the dust in an optically 
thin region. The sudden exposure of the gas-dust 
slab would create a receding evaporation layer where gas is transitioning from 
cold to $10^4-10^6$K hot component \citep[i.e.][]{Dorodnitsyn08b}.
It is not until such an X-ray flux is sufficiently
(i.e. $\tau_{{\rm xw}}\simeq1$) attenuated in the X-ray evaporative
gas/wind, when enough dust can survive, and the UV conversion layer
has a chance to actually settle.
}

The opacity of gas-dust mixture in the UV and IR is dominated by the
opacity of dust, $\kappa_{d}$, with two major contributors: silicon
at lower, and carbon at higher temperatures, the temperature dependence
can be approximately described as

\begin{equation}
\kappa_{d}=\kappa_{0}\left(\frac{T}{T_{{\rm sub}}}\right)^{n}\;{\rm for}\;T<T_{{\rm sub}},
\end{equation}
where we assumed $\kappa_{0}=10-50\;{\rm cm}^{2}\textrm{g}^{-1}$
\citep{Semenov03}, and $n\simeq1-2.$ When $T_{g}\gg T_{s}$, In general,
dust grain sublimation time-scale, $t_{{\rm sub}}$ is very short.
The mass of the dust grain can increase or decrease depending on $\Delta P=p_{{\rm vap}}-p_{i}$,
where $p_{{\rm vap}}$ is the saturation vapor pressure, and $p_{i}$
is the partial pressure of specie, $i$ \citep{Phinney89}. Thus dust grain
sublimation time-scale can be estimated as

\begin{equation}
t_{{\rm sub}}=\frac{m_{{\rm gr}}}{\dot{m}_{{\rm gr}}},\label{eq:DustTimeScale}
\end{equation}

\noindent where $m_{{\rm gr}}$ is the mass of a dust grain, $\dot{m}_{{\rm gr}}=4\pi a^{2}P_{{\rm vap}}\sqrt{\frac{\mu_{i}m}{2\pi kT}}$
is the dust grain mass-loss rate density, $\mu_{i}$ is the molecular
weight, and $m_{u}$ is the atomic mass unit. Since $P_{vap}\sim\exp(-{\rm few}\cdot10^{4}/T)$
it is a very sensitive function of the gas temperature. Corresponding 
time-scale is very short compared to $\su{t}{th},\,\su{t}{dyn}$.
For example,
evaluating \eqref{eq:DustTimeScale} for amorphous silicon dust, it
is just

\begin{equation}
t_{{\rm sub}}({\rm MgFeSiO_{4})}\simeq0.22\,{\rm days\mbox{,}}
\end{equation}

\noindent and $t_{{\rm sub}}\ll t_{{\rm dyn}},\su{t}{th} $ indeed follows.

% \subsection{Solution for the dusty disk: two-layer disk:}

% It was suggested that in presence of dust radiation pressure makes
% such a disk supported disk The first argument towards that the dusty
% radiation pressure supported disk is unstable with respect to convection
% makes connection to the properties of 

% The vertical drop of temperature leads to that at some $0<z<z_{s}$
% dust forms and the opacity rises from

% We want to calculate the structure of an accretion disk between $r_{{\rm c}}$
% and $r_{{\rm s}}$. In this region, near the equatorial plane, the
% disk is always dust-free. Densities there are high and vertical balance
% is entirely provided by the gas pressure. Somewhere between $z=0$
% and $z=H$ where $H$ is the total disk scale height, the temperature
% of the disk drops below $T_{{\rm sub}}$ and opacity rises from $\kappa_{{\rm es}}$
% to $\kappa_{{\rm d}}$ and radiation pressure on dust becomes dominant.
% We approximate such disk as consistent of two layers: layer I - part
% of the disk near the equatorial plane that is dominated by gas pressure,
% $P\simeq P_{g}$ and layer II that is sitting on top of layer I. Layer
% II is dominated by radiation pressure on dust: $P\simeq P_{r}=aT^{4}/3.$

% \subsection{Radiation-supported atmosphere}

% The assumption of efficient convective mixing is translated into a
% $S=const$ condition in the region II of the disk.

% Solving the vertical balance equation between $z_{s}^{+}$ and $z_{b}$
% gives:

% \[
% \rho^{1/3}=\frac{\Omega^{2}}{8K}\left(z_{s}^{2}-z^{2}\right)+(\rho_{s}^{+})^{1/3}
% \]

% \subsection{Radiation transfer:}

% similarly to SS73 The effective surface temperature of the disk follows
% from (\ref{eq:Fvisc}) taking into account its definition $\sigma T_{{\rm eff}}^{4}=F^{+}$
% of $T_{\text{eff}}$, it immediately follows

% \[
% T_{\text{eff}}^{4}=\frac{3}{2ac}\dot{M}\Omega^{2}
% \]

% From (\ref{eq:dEdtau})

% \begin{equation}
% E=3\frac{F}{c}\tau+C,\label{eq:EradEqTransfSolution}
% \end{equation}
% where $C$ is constant that should be found from appropriate boundary
% condition. When $\tau\sim0,$ flux is one-sided: $F\simeq F^{+}=cE(0)/2,$
% and $C=E(0)=2F/c$ giving

% \begin{equation}
% acT^{4}\simeq3F\left(\tau+\frac{2}{3}\right)\simeq3F\tau,\label{eq:PhotEradTransfSolution}
% \end{equation}
% where $\tau\gg1$was assumed at the last equality. The relation for
% the optical depth at which temperature drops from its mid-plane value,
% $T_{c}$ to $T_{{\rm s}}$ follows from (\ref{eq:PhotEradTransfSolution})

% \[
% \tau_{{\rm s}}=\frac{8\pi}{9}\frac{acT_{{\rm s}}^{4}}{\dot{M}\Omega^{2}}
% \]

% Similarly, from ():

% \[
% E_{{\rm c}}\simeq E_{{\rm s}}+3\frac{F}{c}\left(\tau_{{\rm c}}-\tau_{{\rm s}}\right)
% \]

% \section{Vertical structure}

% \[
% -\frac{\pi^{3/2}}{\sqrt{3}\Gamma\left(-\frac{1}{3}\right)\Gamma\left(\frac{11}{6}\right)}\simeq0.84
% \]

% Near the equator

% From ()

% \[
% F=
% \]

% \section{Magnetic field: from galactic scales to the torus}

% \section{Self-gravity}

% The mass $M_{{\rm d}}$ of a standard geometrically-thin accretion
% disk is typically $M_{{\rm d}}\simeq\Sigma R^{2}\ll M_{{\rm BH}}$,
% xwhere $\Sigma(g\,cm^{-2})$ is the disk surface density {[}..Input
% more{]}. The familiar Toomre parameter for the disk with toroidal
% magnetic field can be found from the corresponding dispersion analysis
% {[}Ref{]}:

% \begin{equation}
% Q_{{\rm m}}=\frac{k}{\pi G\Sigma},\label{eq:Qm}
% \end{equation}
% where $C_{F}=(c_{{\rm A}}^{2}+c_{{\rm s}}^{2})^{1/2}$ is the fast
% magnetosonic speed, $c_{{\rm A}}^{2}=B^{2}/(4\pi\rho)$ is the Alfven
% speed, $c_{{\rm s}}^{2}\equiv P/\rho$ is sound speed, and $k$ is
% the epicyclic frequency ($k\simeq\Omega$ for Keplerian disk).

% \section{Disk outflow}

% Due to external illumination, the inflowing geometrically thing accretion
% disk puff-up, becoming convective and forming an outflow from its
% most illuminated side. A particular property of the pc-scale illuminated
% disk is a big mismatch between the energy dissipated locally in the
% accretion disk and external radiation flux. This results in a formation
% of the preheated zone in an accretion disk. Numerical simulations
% is in a qualitative accord with the existence of the preheated zone.
% Immediately before the dust-sublimation surface, the radiation flux
% is directed along the spherical radius ${\bf r}.$ At larger $r$
% the radiation flux ${\bf F}$ still mostly is radial. Beyond dust
% sublimation surface, temperature of the gas drops so that $\nabla\cdot{\bf F}\simeq0$
% in an optically thick radiative zone. Solving for the radiative and
% radial balance in this zone, \citet{Dorodnitsyn11a} derived the following
% relation:

% \begin{equation}
% T=1+\frac{1}{4}\frac{T_{{\rm v}}}{T_{{\rm s}}}\left(1-\Gamma_{{\rm IR}}(\theta)-\frac{\Upsilon^{2}}{\sin^{2}(\theta)}\right)\left(\frac{1}{r}-\frac{1}{r_{0}}\right),\label{eq:T_radial}
% \end{equation}
% where temperature was normalized by $T_{{\rm s}}$. When deriving
% () it was assumed that radiation propagates along the spherical radius
% $r$ and the everything is governed by a simple radial balance equation
% and the equation for $dT/dr$ at fixed inclination $\theta$. The
% parameter $\Upsilon=l/l_{{\rm K}}$ where $l_{{\rm K}}=\sqrt{GM}r$
% is the Keplerian specific angular momentum. Equation () does not make
% any assumptions about the radial distribution of $T$ and $\rho$
% and generalizes corresponding equations of \citet{BKZeld68}derived
% for the spherically-symmetric, non-rotating star. The typical value
% of the $\frac{T_{vg}}{T_{{\rm s}}}\simeq10^{3}$. In order $dT/dr<0,$
% this large factor is compensated by other small factor, $\left(1-\Gamma_{{\rm IR}}(\theta)-\frac{\Upsilon^{2}}{\sin^{2}(\theta)}\right)$
% ,resulting in

% \[
% \Upsilon<\sin(\theta)\sqrt{1-\Gamma_{IR}},
% \]
% that is in the equatorial plane, the torus is sub-Keplerian. There
% likely be a range of parameters when 

% \section{The wind}

% Dynamical and geometrical properties of the torus are set by the competition
% inflow and radiation pressure.

% Whenever numerical simulations include X-rays along with the UV pressure,
% it is found that X-ray evaporation precedes the UV-IR transition layer.
% It is well known that in order to have a finite velocity at infinity,
% an outflow from a potential well, should smoothly pass a critical
% point, i.e. in a point where the speed of the gas equals the speed
% of sound. Correspondingly, along the streamline, in order of increasing
% distance it should be

% \begin{equation}
% \Gamma<1\left.\right|_{r\leqq r_{c}}\;\longrightarrow\:\Gamma=1\left.\right|_{r=r_{c}}\:\longrightarrow\:,\Gamma>1\left.\right|_{r>r_{c}},\label{eq:WindCritPointGammaRelation}
% \end{equation}
% where $r(s)$ is spherical radius following the streamline $s(x,z).$
% That is the transonic flow, i.e. that is the only type of flow which
% has a chance to end up far form the torus, will have a dense base,
% where the gas pressure is not small and, correspondingly, the density
% is relatively high. An important consequence of the torus radiation
% pressure on dust-driven wind hypothesis is that such wind is should
% cross the dust sublimation surface. The base of the wind is thus is
% driven by the thermal pressure, with almost or no dust at all, then
% the projection of the radiation force on the streamline, $\hat{{\bf s}}\cdot{\bf g}_{{\rm rad}}$
% increases, so that at $r_{c}(x,z)$ the condition (\ref{eq:WindCritPointGammaRelation})
% is fulfilled. 
% \begin{description}
% \item [{Question}] 1. Is the the distribution of the radiation pressure
% force inside the torus enough to provide the torus thickness scale,
% $H/r\sim1$ 
% \item [{Question}] 2. 
% \end{description}
% Accretion disk with dust.

% If the obscuring torus would consist of orbiting gas, Virial theorem
% predicts that in order to be geometrically thick at a distance of
% a putative torus, $r\simeq1\,\text{pc},$ gas temperature would be
% of of the order of $10^{6}$K for a $10^{7}M_{\odot}$ .

% WKB7244DF The radiation field near within this region is roughly consisting
% from two parts: remote contribution coming from the inner parts of
% accretion disk, $L_{{\rm disk}}$ and local contribution from accretion
% disk $L_{{\rm loc}}$. The corresponding scale usually associated
% with $L_{UV}$ is

% \begin{equation}
% r_{{\rm sub}}^{{\rm UV}}=0.54\sqrt{L_{46}}T_{1800}^{-2}\,{\rm pc}.\label{eq:RadSub}
% \end{equation}

%{\red 

% \subsection{Magnetic fields, I am working on this section }

% CURRENT WORK 
% \begin{enumerate}
% \item Beyond $r_{{\rm out}}$ the disk is cold and dusty and mostly gas
% pressure supported; magnetic fields providing angular momentum transport
% through magneto-rotational instability. From equipartition arguments
% it follows that beyond $R_{{\rm out}}$ such a small-scale magnetic
% field is relatively weak. 
% \item The dust-driven vertical convection also passively drags small-scale
% magnetic field along with it. It is also possible that an increase
% of the total pressure, $P_{{\rm tot}}=P_{{\rm g}}+\Pi_{{\rm rad}}$
% increases also the strength of the equipartition field and leading
% to correspondingly increased escape of magnetic flux through magnetic
% buoyancy. 
% \end{enumerate}
% Typically, it is assumed that magnetic pressure contributes little
% to the disk thickness. Assuming that galactic gas is sufficiently
% ionized to warrant that the magnetic flux if frozen in the gas element,
% from $\nabla\cdot B=0$ and considering only one-dimensional transport
% of the radial component of magnetic field, $B_{r}$

% \[
% \frac{1}{r^{n}}\partial_{r}(r^{n}B_{r})=0\mbox{{,}}
% \]
% where $n=1$ for cylindrical and n=2 for spherical geometry. Thus,
% flux conservation gives

% \begin{equation}
% B\simeq\left(\frac{r_{{\rm in}}}{r_{{\rm out}}}\right)^{n}\label{eq:B_eq}
% \end{equation}
% Adopting $r_{{\rm out}}=r_{{\rm AGN}}$, from (\ref{eq:rAGN}) and
% arbitrary $r_{in}=1$ pc we obtain from (\ref{eq:B_eq}) $\langle B\rangle_{1{\rm {pc}}}\simeq(40r_{4}-160r_{4}^{2})B_{10\mu G}$.
% Correspondingly, galactic magnetic field can be a viable dynamical
% player in the torus region. }


\subsection{Discussion}
\begin{enumerate}
\item 
The disc mid-plane temperature increases towards smaller $r$ according
to $T_{c}\propto r^{-9/10}\simeq r^{-1}$. At the radius $r_{{\rm out}}$
the mid-plane temperature, $T_{c}$ equals the temperature of dust
sublimation $T_{{\rm sub}}\simeq1500$K. Already
when $T_{c}$ reaches several hundred K, the contribution from radiation
pressure on dust increases. When $T_{c}>\Tsub$ the mid-plane is cleared
from dust and, the total pressure at the mid-plane is dominated by
that of the gas. However just above the mid-plane as the temperature
drops to $T(z)<\Tsub$ opacity increases by two orders of magnitude
and so does the coupling between vertical radiation flux and gas.
\item The local "production-rate" of radiation in a disk depends in the
\textit{local} mass-accretion rate, $\mdt$. For the envelope of a
disk to "puff up" or to become unbound, i.e. to form an outflow,
such disk should have the \textit{local} mass-accretion rate, $\mdt > \mdt[cr]$
of such disk should be super-Eddington as calculated with respect
to the opacity of dust.
\end{enumerate}



\section{Conclusions}

The goal of this paper is to study the vertical structure of a region
in an AGN accretion disk in which local radiation pressure on dust
in important. We have shown that such active dusty region (ADR) is
approximately bounded at the outside by the dust sublimation radius
on the disk mid-plane and on the inside by the dust sublimation radius
at the disk surface. It has been suggested by \citep{CzernyHryniewicz11}
that the pressure of the disk's own, local radiation on dust can drive
large-scale \textquotedbl failed\textquotedbl{} winds. \citet{BaskinLaor2018MNRAS}
recalculated the dust opacity based on the inclusion of the new data
for graphite grains. The latter authors predicted in that in result
of such enhanced opacity, such a disk can \textquotedbl bulge up\textquotedbl{}
and form a compact torus. In this work we focus on a general region
where radiation pressure can impact the vertical structure of the
accretion disk in AGN. As the dusty gas spiral from galactic scales
towards the nucleus it is generally quite cold so that the disk is
very thin. Closer to the BH gas heats up to to internal \textquotedbl viscous\textquotedbl{}
dissipation until such internally generated radiation starts to influence
the vertical structure of the disk via radiation pressure on dust
grains. Our main result is that within a region where the temperature
of the disk, is comparable to the dust sublimation temperature, $T_{{\rm sub}}$
the radiation pressure on dust can have a major effect on the disk
vertical structure and and dynamics: 
\begin{enumerate}
\item Outer boundary of the ``active'' dust region in the disk is approximately
identified as the radius, $r_{{\rm out}}$ where the temperature at
the disk mid-plane equals the dust sublimation temperature, $T_{{\rm sub}}$.
For $M_{{\rm BH}}=10^{7}M_{\odot}$ , $r_{{\rm out}}\simeq0.1$pc.
At $r<r_{{\rm out}}$ dust is cleared near the mid-plane and there
is a dramatic jump of opacity along the vertical through the disk.
The inner boundary of this region is located at the radius where dust
completely disappears inside the disk, i.e. at $T_{s}=T_{{\rm sub}}$,
where $T_{s}$ is the disk surface temperature, Interestingly, AGN
is thus has two regions in accretion disk where radiation pressure
is important: one close to the BH as predicted by the standard SS73
theory, and the other further away at approximately within a region
$r_{{\rm in}}(T_{s}=T_{{\rm sub}})<r<r_{{\rm out}}(T_{c}=T_{{\rm sub}})$. 
\item We argue that mass accretion rate at $r_{{\rm out}}$ need not to
correspond the mass accretion rate devised form the bolometric luminosity
of an AGN, such as Eddington accretion rate: $\dot{M}_{{\rm E}}\simeq0.2 M_{7}$
$\MsolYrM$. 

Local critical accretion rate on the other hand $\dot{M}_{{\rm cr}}\propto1/\kappa_{d}$
and if $\dot{M}_{{\rm cr}}\simeq2.4-10M\odot$, depending on $\kappa_{d}$,
the radiation pressure makes the disk slim (note however that a factor
of about a few is inevitable introduced in these estimates). In our
scenario the ADR internal structure is entirely driven by the the
local accretion rate. It is assumed that all the access gas is removed
by the winds, or participate in large-scale flows. The ability of
the disk to develop geometrically thick atmosphere also depends on
a factor: $q=r_{{\rm out}}/R_{{\rm sub}}$. We argue that $q\gtrsim1$
can be a better proxy for dust surviving above the AD as opposed to
$r(T_{s}=T_{{\rm sub}})/R_{{\rm sub}}$adopted in previous works. 
\item We found that ADR is strongly convectively unstable with significant
vertical energy transport via convection. Convection results in effective
cooling of the disk interior. We also argue that in the context of
the BLR-ADR connection, convection provides the turbulence driver
for the BLR. 
\end{enumerate}

\section{APPENDIX}

\subsection{Convective disk}

Here we assume that energy is transported towards the surface of a
disk via convection. That is $F_{{\rm conv}}\gg F_{{\rm rad}}$ where
$F_{{\rm conv}}$ and $F_{{\rm rad}}$ are convective and radiation
fluxes respectively c.f.. Convection tends to establish isoentropic
distribution: $S=const.$ where $S$ is found from (\ref{eq:EntropyRad}).
In layer II radiation pressure dominates and we adopt polytropic equation
of state for radiation \citet{BKBlinn77}:

\begin{equation}
P=K\rho^{4/3}\mbox{,}\label{eq:P=Kro4/3}
\end{equation}
where

\begin{equation}
K=\left(\frac{3S^{4}}{256a}\right)^{1/3}\simeq const.\label{eq:Kpolitrrad}
\end{equation}
is constant. For polytropic e.s. (\ref{eq:P=Kro4/3})
equation (\ref{eq:dPdz}) reads:

\begin{equation}
\frac{dP}{dz}=-\Omega^{2}K^{-3/4}zP^{3/4}\mbox{,}\label{eq:dPdZAdiab}
\end{equation}

Solving vertical balance equation (\ref{eq:dPdz}) with (\ref{eq:P=Kro4/3})
and (\ref{eq:Kpolitrrad}) it is possible to obtain the following
simple relations valid in layer II (see Appendix):

\begin{eqnarray}
\rho & \simeq & \rho^{+}\left(1-\frac{z^{2}}{z_{b}}\right)^{3}\mbox{,}\label{eq:roSol}\\
P & \simeq & P^{+}\left(1-\frac{z^{2}}{z_{b}}\right)^{4}\mbox{,}\label{eq:PresSol}
\end{eqnarray}
where we for simplicity we took into account that $P_{b}\ll P^{+}$,
$\rho_{b}\ll\rho^{+}$ and adopt $P_{b}\simeq0$, $\rho_{b}\simeq0,$
at $z=z_{b},$ when simplifying (\ref{eq:roSol}),(\ref{eq:PresSol}).
Equation (\ref{eq:roSol}) is integrated to obtain surface density
of layer II:

\[
\sigma_{2}=\int_{z_{s}}^{z_{b}}\rho\:dz\simeq\frac{16}{35}\rho^{+}z_{b},
\]
where $z_{s}\simeq0$ since $z_{s}\ll z_{b}$, and the numerical coefficient
only weakly depends on the simplifying assumptions such as those implied
in the derivation of (\ref{eq:roSol}),(\ref{eq:PresSol}).

The characteristic scale-height of a flat disk is

\[
H=\frac{C}{\Omega},
\]
where $C$ is the characteristic speed of sound:

\[
C_{b}^{2}=\frac{2HT^{4}}{3\Sigma}+{\cal R}T
\]

Thus the gas supported layer has thickness:

\[
h_{g}\simeq4.40\times10^{-4}\left(T_{3}\right)^{1/2}\left(\frac{M_{7}}{R_{0.1}^{3}}\right)^{-1/2}\text{pc},
\]
i.e. gas-supported layer is extremely thin.

\[
\bar{\nu}\Sigma=\text{\ensuremath{\frac{\dot{M}}{3\pi}}J},
\]

where $\bar{\nu}_{1}=\Sigma_{c}^{-1}\int_{-h}^{h}\rho\nu\,dz$ and
approximately $\bar{\nu}\simeq\nu$

\[
\dot{M}=3\pi\Sigma\bar{\nu}_{c}.
\]

\[
T_{1}^{4}=\frac{9}{16\sigma}\Omega^{2}\text{(}\Sigma_{c}\bar{\nu}_{c}+\frac{\nu_{c}}{\kappa_{c}}\text{)}.
\]

\[
\rho=\frac{\Sigma}{2H}
\]

\[
\sigma=\int_{z}^{\infty}\rho\:dz,
\]

Solving (\ref{eq:dPdZAdiab}) with $P_{s}=P(z_{s})=P^{+}=P^{-}$,
% (note (\ref{eq:Ps=P-})) and \ref{eq:{Ps}}
$P(z_{b})=P^{+}=P^{-}$

\begin{equation}
P^{1/4}=P_{s}^{1/4}\left[1-\left(1-\left(\frac{P_{b}}{P_{s}}\right)^{1/4}\right)\frac{z_{s}^{2}/z_{b}^{2}-z^{2}/z_{b}^{2}}{z_{s}^{2}/z_{b}^{2}-1}\right]\mbox{.}\label{eq:PresSolTot}
\end{equation}

when $P_{b}\ll P_{s}$, $\rho_{b}\ll\rho^{+}$, equation (\ref{eq:PresSolTot})
is simplified to (\ref{eq:PresSol}).

where $K$ is found from (\ref{eq:Kpolitrrad}) and assumed constant.
% (see discussion \ref{Convective-layer})


\bibliography{BibListAGN}

\end{document}

%%% Local Variables:
%%% mode: latex
%%% TeX-master: t
%%% End:
